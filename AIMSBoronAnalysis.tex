\documentclass[11pt,a4paper,final]{article}
\usepackage[latin1]{inputenc}
\usepackage{amsmath}
\usepackage{amsfonts}
\usepackage{amssymb}
\author{H.S. Barnard, Z.S. Hartwig, B.N. Sorbom, D.G.Whyte}
\title{In-Situ Spatially Resolved Measurement of Boron Erosion and Deposition in the Alcator C-Mod Tokamak}
\begin{document}
\maketitle
\section{Introduction}
\begin{itemize}
	\item All current tokamaks with High-Z PFCs require low-Z conditioning to prevent core impurities and achieve $H_{98}\sim 2$
	\item Degradation of plasma performance appears to be correlated plasma exposure time and localized erosion of B outside of the divertor.
	\item Low-Z coatings degrade too quickly for long pulse devices and cannot be applied in steady state devices.
	\item Low-Z coatings are shown to work well, but have never been well characterized, especially on the relevant timescales of their erosion.
	\item Boron erosion is an important proxy for studying the formation of high-Z sources 
\end{itemize}

\section{Experiment}
\begin{itemize}
	\item Brief description of AIMS
	\item Photo peak used for B detection
	\item should be able to provide shot to shot 
	\item Neutron reaction used for B detection
\end{itemize}

\section{AIMS Boron Measurements}
...
\section{Ex-situ Validation}
\begin{itemize}
	\item External PIGE analysis
	\item 
\end{itemize}
\section{Discussion}
\begin{itemize}
	\item First ever in-situ spatially resolved measurement of Boron
	\item Issues with calibration drift, blurred peaks
	\item Neutron data
	\item More gamma detectors closer for better statistics
\end{itemize}
\end{document}