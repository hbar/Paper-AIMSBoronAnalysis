%\documentclass[11pt,a4paper,twocolumn]{elsarticle}
%\documentclass[preprint,12pt]{elsarticle}

%% Use the option review to obtain double line spacing
%% \documentclass[preprint,review,12pt]{elsarticle}

%% Use the options 1p,twocolumn; 3p; 3p,twocolumn; 5p; or 5p,twocolumn
%% for a journal layout:
%% \documentclass[final,1p,times]{elsarticle}
%% \documentclass[final,1p,times,twocolumn]{elsarticle}
%% \documentclass[final,3p,times]{elsarticle}
\documentclass[final,3p,times,twocolumn]{elsarticle}
%% \documentclass[final,5p,times]{elsarticle}
%% \documentclass[final,5p,times,twocolumn]{elsarticle}

\usepackage[latin1]{inputenc}
\usepackage{amsmath}
\usepackage{amsfonts}
\usepackage{amssymb}
\usepackage{graphicx}
\usepackage{cite}
\usepackage[usenames,dvipsnames]{color}

\journal{Nuclear Nuclear Fusion}
\begin{document}
%\maketitle
\begin{frontmatter}

\title{In-Situ Spatially Resolved Measurement of Boron Erosion and Deposition in the Alcator C-Mod Tokamak}
%% use optional labels to link authors explicitly to addresses:
%% \author[label1,label2]{<author name>}
%% \address[label1]{<address>}
%% \address[label2]{<address>}

\author{H.S. Barnard, Z.S. Hartwig, B.N. Sorbom, D.G.Whyte}

%\abstract{The accelerator based in-situ materials serveilance diagnostic (AIMS) was recently installed in 2012 to demonstrate the novel application of ion beam analysis to plasma materials interaction (PMI) science in the Alcator C-Mod Tokamak.  The first in-situ, spatially and temporally resolved measurements of material surface erosion and deposition are presented.  }

\begin{abstract}

The Accelerator Based In-situ Materials Surveillance (AIMS) diagnostic was recently developed to demonstrate the novel application of ion beam analysis (IBA) to in-vessel studies of plasma materials interaction in Alcator C-Mod. The AIMS diagnostic injects a 900 keV deuterium ion beam into the tokamak's vacuum vessel between plasma discharges while magnetic fields are used to steer the ion beam to plasma facing component (PFC) surfaces. Spectroscopic analysis of neutrons and gamma rays from the induced nuclear reactions provides a quantitative, spatially resolved map of the PFC surface composition that includes boron (B) and deuterium (D) content. Since AIMS is sensitive to low-Z elements and C-Mod regularly boronizes PFCs, the evolution of B and D on PFCs can be used to directly study erosion, deposition, and fuel retention in response to plasma operations and wall condition processes. AIMS analysis of 18 lower single null I-mode discharges show a net boron deposition rate of $6\pm2$~nm/s on the inner wall while subsequent inner wall limited discharges and a disruption did not show significant changes in B. Measurements of D content showed relative changes of $>$2.5 following a similar trend. This suggests high D retention rates and net B deposition rates of $\sim$18 cm/year of plasma exposure are possible and depend strongly on the plasma conditions. Ex-situ IBA was also performed on the same PFCs after removal from C-Mod, successfully validating the AIMS technique. These IBA measurements also show that the B content on the inner wall varied toroidally and poloidally from 0-3000 nm, demonstrating the importance of the spatial resolution provided by AIMS and the sensitivity of PFCs to B-field alignment.

\end{abstract}

\begin{keyword}
%% keywords here, in the form: keyword \sep keyword

%% MSC codes here, in the form: \MSC code \sep code
%% or \MSC[2008] code \sep code (2000 is the default)

\end{keyword}

\end{frontmatter}

%==============================================================================
% Introduction
%==============================================================================

\section{Introduction}
\label{sec:Intro}
Plasma materials interactions (PMI) present one the most significant engineering challenges for reactor-scale fusion devices. PMI however are difficult to study in laboratory tokamaks because they typically operate in short pulses with varied plasma conditions and provide little or no access to plasma facing components (PFC) under normal operating conditions. Due to this experimental constraint, it is often impossible to correlate plasma conditions to the effects of PMI thus leaving the dynamic changes in fusion materials severely understudied.  The recent development of the AIMS diagnostic \cite{RSIPaper} (section \ref{sec:AIMSOverview}), however, has shown that it is now possible to directly measure the retention of hydrogen isotopes from the plasma \cite{HartwigDRetention} and the erosion and deposition of low-Z materials (presented in section \ref{sec:BZNOverview}). These measurements provide new insight into the timescales and spatial variation of PMI effects on PFCs. 

%==============================================================================
% Background on IBA and AIMS
%==============================================================================

\section{Ion beam analysis and AIMS}
\label{sec:AIMSOverview}

A variety of ion beam analysis (IBA) techniques are commonly used to perform \textit{ex-situ} analysis on PFCs after they are removed from the tokamak. IBA involves using ion beams to induce nuclear reactions, elastic collisions, or atomic excitations in materials.  Spectroscopic measurements of the reaction products are then used to identify and quantify isotopes in the material's surface. A comprehensive description of these established methods  can be found in \cite{tesmer1995handbook}.

These IBA techniques serve an important function for studying long term aggregate PMI effects on PFC \cite{wright2011plasma} and were instrumental in validating the AIMS technique (section \ref{sec:PIGE}). However, there are only rare occasions when PFCs can be removed for analysis, IBA studies are typically limited to long term net effects of PMI, often over the course of hundreds to thousands of plasma discharges.  Since these timescales spanning months of an experimental campaign with varying plasma configurations, there is essentially no way to diagnose how one plasma configuration affects PFCs as compared to another.

\begin{figure*}[h]
 \centering
  \includegraphics[width=140mm]{figures/AIMS_Overview_Schematic.png}
 \caption{Left: CAD model of the AIMS diagnostic installed on Alcator C-Mod.  Right: Schematic of AIMS components.  AIMS utilizes a radio frequency quadrupole (RFQ) accelerator produce a 900 keV D$^+$ beam to induce nuclear reactions on the surface of plasma facing components (PFC). Spectroscopy of the resulting neutrons and gamma rays allow for the identification and quantification of isotopes on PFC surfaces. AIMS uses beam optics and toroidal field $B_\phi$ to steer the beam and achive spatially resolved measurements.}
 \label{fig:AIMSOverviewSchematic0}
\end{figure*}

A new diagnostic technique referred to as accelerator-based in situ materials surveillance (AIMS) was developed to simultaneously improve the spatial and temporal resolution of IBA measurements. AIMS utilizes a compact linear accelerator, gamma detectors, and neutron detectors to measure the evolution of PFC surface composition inside a magnetic confinement device. The technique is non-destructive to the PFCs, can access large fractions of the total PFC surface area, and is not disruptive to facility operations because it is designed to operate between plasma discharges. AIMS is essentially an IBA based in-situ diagnostic that is directly integrated with a tokamak. In addition, AIMS also provides spatially resolved measurements with approximately 2 cm spatial resolution on the shot-to-shot time scale by using the tokamak's magnetic field coils and DC supplies to steer and target the beam over a relatively large region of the plasma facing first wall \cite{RSIPaper}.  The development and implementation of the AIMS diagnostic, illustrated in figure \ref{fig:AIMSOverviewSchematic0}, was successfully implemented on Alcator C-Mod, making its first measurements during the 2012 experimental campaign. These measurements of boron erosion are presented in section (???????).

%==============================================================================
% Background on Boron and low Z erosion
%==============================================================================

\section{Boronization and Erosion}
\label{sec:BZNOverview}

High-Z refractory metal PFCs, particularly tungsten, are leading material choices for reactor PFCs. This is due to their low erosion, low tritium retention, favorable thermal properties, and robustness to neutron damage and activation \cite{WWoBoronization}. As a result, Alcator C-Mod uses almost entirely molybdenum PFCs -- a high-Z refractory metal with similar properties to W.  Despite the advantages of W and Mo, when high-Z elements enter the plasma energy confinement and plasma performance is degraded severely due to increased radiation from the high-Z impurities. %This tends to have strong negative impact on C-Mod plasma performance when operating with bare Mo PFCs, typically only achieving H-Mode plasmas with poor confinement $H_\mathrm{ITER,89} \sim 1$.  
%The impurity problems were successfully mitigated early on with the C-Mod boronization process. 
This issue has been resolved in C-Mod with the boronization process which involves plasma depositing boron on PFCs using an electron cyclotron discharge cleaning (ECDC) plasmas with a helium-diborane gas mixture. The plasma is created near the ECDC resonance which is swept in the the radial direction by varying the toroidal field to distribute the boron across PFCs in the divertor. 

When PFCs are plasma coated with boron using this procedure, the impurity radiation of the confined plasma drops substantially and the performance improves, essentially doubling energy confinement time, with $H_\mathrm{ITER,89}$ approaching 2.% The consistent use of boroniation means that there is 
The improvement in performance degrades with plasma exposure yet boron remains on most surfaces according ex-situ measurements. This indicates that the decrease in performance is due to localized erosion of boron that occurs over a relatively small area of the PFCs (with typical parameters: f = 2.45 GHz, B = 0.088 T at $R_o = 0.67$ m). (%10\% B$_2$D$_6$, 90\% He) \cite{WWoBoronization}.
% (with typical parameters: field f = 2.45 GHz, B = 0.088 T at $R_o = 0.67$ m). (%10\% B$_2$D$_6$, 90\% He),

Even though boronization is such a successful technique for improving plasma performance through impurity control, there is still relatively little quantitative understanding of its distribution after boronization and the erosion patterns that are produced with plasma exposure. Boronization and other low-Z coatings are not feasible for long pulsed devices or reactors, however, the erosion and deposition of boron is important to study because these regions lead to impurity ?????????? injection in high performance tokamaks like C-Mod can identify the regions or plasma conditions that are responsible for the degraded performance.  Thus, measurements of boron can serve as a proxy for high-Z erosion and may play a vital role in understanding erosion in reactors as well as the complex net transport of materials throughout the the tokamak due to PMI.


\begin{figure*}[p]
 \centering
  \includegraphics[width=125mm]{figures/TileMap_Combined_Highlighted.pdf}
 \caption{\small Top: Tile map of the C-Mod inner wall, upper divertor, and lower divertor EF-1 shelf, toroidally spanning the A,B, and C port regions. Bottom: Zoomed-in tile map showing the locations of the PIGE and AIMS measurements. Top Right: Deuteron beam trajectories are shown for four trajectories spanning the range of the AIMS measurements. The tiles highlighted in yellow/red were removed and PIGE analyzed following AIMS measurements.  The green ellipses indicate the calculated location and beamspots for AIMS measurements based on modeling. The filled ellipses represent the four locations used for the majority of the AIMS measurements corresponding to toroidal beam steering fields $B_\phi$ = \{0.000, 0.0582, 0.1135, 0.1618\} Tesla (in order from top to bottom).}
 \label{fig:TileMap0}
\end{figure*}

%==============================================================================
%
%==============================================================================

\section{AIMS analysis of boron in Alcator C-Mod}

AIMS Measurements were made at 4 poloidal locations on the inner wall of Alcator C-Mod using the AIMS technique before, during, and after the last run day of the 2012 C-Mod campaign to observe the effects of plasma discharges. Subsequent measurements were taken during the months following the campaign to observe changes due to plasma conditioning operations that include boronization, electron cyclotron discharge cleaning (ECDC), and glow discharge cleaning (GDC).  


The AIMS geometry and timeline are presented in figure~\ref{fig:TileMap0} and section~\ref{sec:AIMSPIGETimeline}.



\section{Ion Beam Analysis for AIMS Validation}

\section{AIMS Results and Discussion}

\section{Summary and Conclusions}

The AIMS diagnostic was successfully implemented on Alcator C-Mod yielding the first spatial resolved and quantitative in-situ measurements of boron in a tokamak.  By combining AIMS neutron and gamma measurements, time resolved and spatially resolved measurements of boron were made, spanning the entire AIMS run campaign which included lower single null plasma shots, inboard limited plasma shots, a disruption, and C-Mod wall conditioning procedures.  These measurements demonstrated the capability if AIMS to perform inter-shot measurements at a single location and spatially resolved measurements on over longer timescales (with great potential for improved timescales and resolution).  This demonstration showed the first in-situ measurements of surfaces in a magnetic fusion device with spatial and temporal resolution which constitutes a major step forward in fusion PMI science.

An external ion beam system was also implemented to perform ex-situ ion beam analysis (IBA) on large components removed from Alcator C-Mod.  This system was used to perform particle induced gamma emission (PIGE), a well established IBA technique, on tile modules to validate the AIMS technique.  From these external PIGE measurements, a spatially resolved map of boron areal density was constructed for a section of C-Mod inner wall tiles that overlapped with the AIMS measurement locations.  These measurements showed the complexity of the poloidal and toroidal variation of boron areal density between PFC tiles on the inner wall ranging from 0 to 3$\mu$m of boron.  Using these well characterized ex-situ measurements to corroborate the in-situ measurements, AIMS showed reasonable agreement with PIGE, thus validating the quantitative boron detection capability of the AIMS technique.


\bibliography{AIMSBoronAnalysis.bib}{}
\bibliographystyle{plain}

\end{document}
